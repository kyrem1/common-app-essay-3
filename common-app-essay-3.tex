\documentclass[12pt]{article}

%%%%%%%%%%%%%%%%%%%%%%%%%%%%%%%%%%-- Settings --%%%%%%%%%%%%%%%%%%%%%%%%%%%%%%%%%%%%%%%%%%%
\usepackage[english]{babel}

% - Margin - 1 inch on all sides
\usepackage[letterpaper]{geometry}
\usepackage[utf8]{inputenc}
\geometry{top=1.0in, bottom=1.0in, left=1.0in, right=1.0in}

% - Double Spacing -
\usepackage{setspace}
\doublespacing
\setstretch{2.00}

% % FancyHDR settings
% \setlength{\headheight}{15.2pt}
% \usepackage{fancyhdr}
% \pagestyle{fancy}
% \fancyhf{}

% Indent First paragraph
\usepackage{indentfirst}

% Other Packages
\usepackage{outlines}

% Title Settings
\title{
    \vspace{2in}
    \textmd{\textbf{National Merit Essay}}\\
    \normalsize\vspace{0.1in}\small{Due\ October\ 7th,\ 2020}\\
    \vspace{3in}
}
\author{James Harbour}

\renewcommand{\footnoterule}{%
  \kern -3pt
  \hrule width \textwidth height 0.5pt
  \kern 2pt
}

%%%%%%%%%%%%%%%%%%%%%%%%%%%%%%%%%%-- Assignment --%%%%%%%%%%%%%%%%%%%%%%%%%%%%%%%%%%%%%%%%%%%

  % Common Application : general essay
  % Characters Min: 0
  % Characters Max: 3500
  % Words Max: Approximately 750

%%%%%%%%%%%%%%%%%%%%%%%%%%%%%%%%%%-- TODOLIST --%%%%%%%%%%%%%%%%%%%%%%%%%%%%%%%%%%%%%%%%%%%

%%%%%%%%%%%%%%%%%%%%%%%%%%%%%%%%%%-- Document --%%%%%%%%%%%%%%%%%%%%%%%%%%%%%%%%%%%%%%%%%%%

% Begin Document
\begin{document}
\maketitle
\pagebreak
 \begin{center}
   \emph{Of Pen lights and Headlights}
 \end{center}
 Prompt: \emph{To help the reviewers get to know you, describe an experience you have had, a person who has influenced you, or an obstacle you have overcome. Explain why this is meaningful to you. Use your own words and limit your response to the space provided on the application.}
\raggedright\setlength{\parindent}{0.5in}

To study mathematics is to brandish a pen light against the darkness of the abyss. Hence, it is not surprising that few have the stomach for the matter; the continual ascertainment that one understands nothing presents too grim a fate for most. To the remaining few---triumphant fools who somehow can maintain sanity in the face of a constant feeling of stupidity---the subject grants a paradoxical chance to see truth in its purest visage: an art form. Beyond the everyday struggles of studying the truths of the ancient giants, there arrives moments of clarity where one experiences the unrelenting joy of a genuine triumph over the abyss. The purity of this happiness stands uncontested as the experience takes place inside one's own mind, cold and humble without the taint of external influences.

I am now one of these triumphant fools---an acolyte of that crazed cult whose only desire is truth.

My first collegiate pure mathematics course, ``MGF 3301 : Bridge to Abstract Mathematics,'' began with the rather peculiar words: ``You will not learn any new material, but this will be the hardest class you have ever taken.'' Stated more explicitly, the content of the class does not traverse beyond the mathematical knowledge of an average middle-schooler. From this limited information, one likely would conclude that the course lacks anything of interest; however, when I heard the first words of my professor, I smiled. I knew what I was getting into: something finally interesting. This description begs the question: what was the difference between this course and middle school? Mathematical proof i.e. exhaustive deductive reasoning. Compared to the fuzzy, hand-waving arguments utilized in primary school, real mathematics paints a much clearer, more complete picture. In real mathematics, every statement must be justified, every argument complete. This class was my first introduction to an unscrupulous way of looking at the world that cemented my passion for the activities of the mind.

Every weekday, after my morning high school classes ended, I made my long commute to the local college. During those commutes, instead of uselessly obsessing over the gaping divide between the quality of my university courses and that of my supposedly ``rigorous'' high school classes, I questioned the universe. I spent those drives thinking about mathematics, facing my understanding with skepticism, solving, pondering, and mostly failing to understand the idea that was the subject of the drive that day.

The experience of MGF:3301 extended beyond just the lectures, beyond the studying and the homework; the most important part of the course took place during those long drives. It was during those drives that I evolved from an enrapt spectator to an actual practitioner of the philosophy ``question everything.'' They taught me the satisfaction of incremental growth---of reaching ever so slightly beyond my intellectual capacity each day. Those long drives through the abyss, alit by only dim headlights, introduced me to the beauty of uncertainty.

Nowadays, I am excited by the prospect of how little I know. Upon hearing that I am taking graduate mathematics courses, the mathematically uninclined pronounce me a ``genius.'' Although their praises are well-meaning, I take care to not let them spoil my perspective of that which I do not know. That sublime feeling of complete uncertainty and that inexorable desire to shine even a pen light into the abyss, both are too precious to me. Through the long car rides spent in their company, these two conflicting ideals have become my guiding principles, my rock, my foundation. In the crazed cult that I have joined, foundations are everything.

\end{document}
